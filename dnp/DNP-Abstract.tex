\documentclass[11pt]{article}
%\markright{Official Use Only}
\usepackage{fancyhdr}
\usepackage{graphicx}
\usepackage{listings}
\usepackage{color}
\usepackage{enumitem}
\usepackage{hyperref}
\usepackage{verbatim}

\textwidth 7.0in
\textheight 9in
%\itemsep 0pt
%\parsep 3pt
%\parindent 5pt
\parskip 3pt
\hoffset -1.in
\voffset -0.5in

\pagestyle{fancy}
\fancyhf{}
\lhead{DNP 2017 Abstract}
\rhead{Augest 2017}

%\usepackage{pdfsync}
\begin{document}
% Figure commands (not needed here)
\renewcommand{\textfraction}{0.1}
\renewcommand{\topfraction}{0.9}
\renewcommand{\bottomfraction}{0.9}
\renewcommand{\floatpagefraction}{0.1}

\begin{center}
{\Large \bf Understanding Uncertainties and Biases in Jet Quenching in High-Energy Nucleus-Nucleus Collisions}\\
\bigskip
LLNL-ABS-738296
\footnote{This work was performed under the auspices of the U.S. Department of Energy by Lawrence Livermore National Laboratory under Contract DE-AC52-07NA27344.}
\\
\bigskip
Matthias Heinz, Ron Soltz, and Aaron Angerami
\end{center}

\begin{abstract}
Jets are the collimated streams of particles resulting from hard scattering in the initial state of high-energy collisions. In heavy-ion collisions, jets interact with the quark-gluon plasma (QGP) before freezeout, providing a probe into the internal structure and properties of the QGP. In order to study jets, background must be subtracted from the measured event, potentially introducing a bias. We aim to understand and quantify this subtraction bias. PYTHIA, a library to simulate pure jet events, is used to simulate a model for a signature with one pure jet (a photon) and one quenched jet, where all quenched particle momenta are reduced by a user-defined constant fraction. Background for the event is simulated using multiplicity values generated by the TRENTO initial state model of heavy-ion collisions fed into a thermal model consisting of a 3-dimensional Boltzmann distribution for particle types and momenta. Data from the simulated events is used to train a statistical model, which computes a posterior distribution of the quench factor for a data set. The model was tested first on pure jet events and then on full events including the background. This model will allow for a quantitative determination of biases induced by various methods of background subtraction.
\end{abstract}

\end{document}